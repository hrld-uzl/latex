%!TEX root = thesis.tex

\chapter{Einleitung}
Ein Spülbecken (ugs. auch Spüle) (in Österreich auch „die Abwasch“, in der Schweiz vorwiegend „Abwaschbecken“), ist meist in die Platte der Küchenarbeitsfläche eingelassen und wird zur Vorbereitung von Speisen (z. B. Reinigen von Obst, Salat oder Gemüse) und zur Säuberung von Geschirr und Küchenmaterialien benutzt. Der Unterschied zum Waschbecken ist hygienischer Natur, während umgangssprachlich diese häufig nicht mehr unterschieden werden. Während in Waschbecken beispielsweise auch mal Schmierstoffe verwendet werden, ist es eine subjektive Frage inwieweit man diese gemeinsam mit Lebensmitteln bearbeiten möchte. Schüttsteine gab es bereits im Mittelalter, zumeist jedoch nur in Burgen oder Schlössern. Der umgangssprachlich in manchen Regionen (z. B. dem Rheinland) übliche Ausdruck Spülstein für das Spülbecken stammt noch hiervon ab.

Das Spülbecken zählt gemeinsam mit Herd und Kühlschrank zur Grundausstattung einer modernen Küche. Die Spüle ist zumeist mit einer Küchenarmatur ausgestattet und somit der zentrale Punkt, um die Küche mit heißem und kaltem Wasser zu versorgen. Im fränkischen Sprachraum wird das zweite Becken auch als Abfleibecken bezeichnet.

