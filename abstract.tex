%!TEX root = thesis.tex

\cleardoublepage
\thispagestyle{plain}

\pdfbookmark{Spülbecken}{Spülbecken}
\paragraph{Kurzfassung}
Ein Spülbecken (ugs. auch Spüle) (in Österreich auch „die Abwasch“, in der Schweiz vorwiegend „Abwaschbecken“), ist meist in die Platte der Küchenarbeitsfläche eingelassen und wird zur Vorbereitung von Speisen (z. B. Reinigen von Obst, Salat oder Gemüse) und zur Säuberung von Geschirr und Küchenmaterialien benutzt. Der Unterschied zum Waschbecken ist hygienischer Natur, während umgangssprachlich diese häufig nicht mehr unterschieden werden.


\cleardoublepage
\thispagestyle{plain}

\foreignlanguage{english}{%
\pdfbookmark{Abstract}{abstract}
\paragraph{Abstract}
A sink—also known by other names including sinker, washbowl, hand basin and wash basin—is a bowl-shaped plumbing fixture used for washing hands, dishwashing, and other purposes. Sinks have taps (faucets) that supply hot and cold water and may include a spray feature to be used for faster rinsing. They also include a drain to remove used water; this drain may itself include a strainer and/or shut-off device and an overflow-prevention device. Sinks may also have an integrated soap dispenser.
}
